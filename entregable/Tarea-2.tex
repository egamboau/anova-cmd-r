\documentclass[]{article}
\usepackage{lmodern}
\usepackage{amssymb,amsmath}
\usepackage{ifxetex,ifluatex}
\usepackage{fixltx2e} % provides \textsubscript
\ifnum 0\ifxetex 1\fi\ifluatex 1\fi=0 % if pdftex
  \usepackage[T1]{fontenc}
  \usepackage[utf8]{inputenc}
\else % if luatex or xelatex
  \ifxetex
    \usepackage{mathspec}
  \else
    \usepackage{fontspec}
  \fi
  \defaultfontfeatures{Ligatures=TeX,Scale=MatchLowercase}
\fi
% use upquote if available, for straight quotes in verbatim environments
\IfFileExists{upquote.sty}{\usepackage{upquote}}{}
% use microtype if available
\IfFileExists{microtype.sty}{%
\usepackage{microtype}
\UseMicrotypeSet[protrusion]{basicmath} % disable protrusion for tt fonts
}{}
\usepackage[margin=1in]{geometry}
\usepackage{hyperref}
\hypersetup{unicode=true,
            pdftitle={Tarea 2},
            pdfauthor={Eduardo Gamboa; Manuel Marín; Megan Alexa Mora; Francisco Sánchez},
            pdfborder={0 0 0},
            breaklinks=true}
\urlstyle{same}  % don't use monospace font for urls
\usepackage{color}
\usepackage{fancyvrb}
\newcommand{\VerbBar}{|}
\newcommand{\VERB}{\Verb[commandchars=\\\{\}]}
\DefineVerbatimEnvironment{Highlighting}{Verbatim}{commandchars=\\\{\}}
% Add ',fontsize=\small' for more characters per line
\usepackage{framed}
\definecolor{shadecolor}{RGB}{248,248,248}
\newenvironment{Shaded}{\begin{snugshade}}{\end{snugshade}}
\newcommand{\KeywordTok}[1]{\textcolor[rgb]{0.13,0.29,0.53}{\textbf{#1}}}
\newcommand{\DataTypeTok}[1]{\textcolor[rgb]{0.13,0.29,0.53}{#1}}
\newcommand{\DecValTok}[1]{\textcolor[rgb]{0.00,0.00,0.81}{#1}}
\newcommand{\BaseNTok}[1]{\textcolor[rgb]{0.00,0.00,0.81}{#1}}
\newcommand{\FloatTok}[1]{\textcolor[rgb]{0.00,0.00,0.81}{#1}}
\newcommand{\ConstantTok}[1]{\textcolor[rgb]{0.00,0.00,0.00}{#1}}
\newcommand{\CharTok}[1]{\textcolor[rgb]{0.31,0.60,0.02}{#1}}
\newcommand{\SpecialCharTok}[1]{\textcolor[rgb]{0.00,0.00,0.00}{#1}}
\newcommand{\StringTok}[1]{\textcolor[rgb]{0.31,0.60,0.02}{#1}}
\newcommand{\VerbatimStringTok}[1]{\textcolor[rgb]{0.31,0.60,0.02}{#1}}
\newcommand{\SpecialStringTok}[1]{\textcolor[rgb]{0.31,0.60,0.02}{#1}}
\newcommand{\ImportTok}[1]{#1}
\newcommand{\CommentTok}[1]{\textcolor[rgb]{0.56,0.35,0.01}{\textit{#1}}}
\newcommand{\DocumentationTok}[1]{\textcolor[rgb]{0.56,0.35,0.01}{\textbf{\textit{#1}}}}
\newcommand{\AnnotationTok}[1]{\textcolor[rgb]{0.56,0.35,0.01}{\textbf{\textit{#1}}}}
\newcommand{\CommentVarTok}[1]{\textcolor[rgb]{0.56,0.35,0.01}{\textbf{\textit{#1}}}}
\newcommand{\OtherTok}[1]{\textcolor[rgb]{0.56,0.35,0.01}{#1}}
\newcommand{\FunctionTok}[1]{\textcolor[rgb]{0.00,0.00,0.00}{#1}}
\newcommand{\VariableTok}[1]{\textcolor[rgb]{0.00,0.00,0.00}{#1}}
\newcommand{\ControlFlowTok}[1]{\textcolor[rgb]{0.13,0.29,0.53}{\textbf{#1}}}
\newcommand{\OperatorTok}[1]{\textcolor[rgb]{0.81,0.36,0.00}{\textbf{#1}}}
\newcommand{\BuiltInTok}[1]{#1}
\newcommand{\ExtensionTok}[1]{#1}
\newcommand{\PreprocessorTok}[1]{\textcolor[rgb]{0.56,0.35,0.01}{\textit{#1}}}
\newcommand{\AttributeTok}[1]{\textcolor[rgb]{0.77,0.63,0.00}{#1}}
\newcommand{\RegionMarkerTok}[1]{#1}
\newcommand{\InformationTok}[1]{\textcolor[rgb]{0.56,0.35,0.01}{\textbf{\textit{#1}}}}
\newcommand{\WarningTok}[1]{\textcolor[rgb]{0.56,0.35,0.01}{\textbf{\textit{#1}}}}
\newcommand{\AlertTok}[1]{\textcolor[rgb]{0.94,0.16,0.16}{#1}}
\newcommand{\ErrorTok}[1]{\textcolor[rgb]{0.64,0.00,0.00}{\textbf{#1}}}
\newcommand{\NormalTok}[1]{#1}
\usepackage{graphicx,grffile}
\makeatletter
\def\maxwidth{\ifdim\Gin@nat@width>\linewidth\linewidth\else\Gin@nat@width\fi}
\def\maxheight{\ifdim\Gin@nat@height>\textheight\textheight\else\Gin@nat@height\fi}
\makeatother
% Scale images if necessary, so that they will not overflow the page
% margins by default, and it is still possible to overwrite the defaults
% using explicit options in \includegraphics[width, height, ...]{}
\setkeys{Gin}{width=\maxwidth,height=\maxheight,keepaspectratio}
\IfFileExists{parskip.sty}{%
\usepackage{parskip}
}{% else
\setlength{\parindent}{0pt}
\setlength{\parskip}{6pt plus 2pt minus 1pt}
}
\setlength{\emergencystretch}{3em}  % prevent overfull lines
\providecommand{\tightlist}{%
  \setlength{\itemsep}{0pt}\setlength{\parskip}{0pt}}
\setcounter{secnumdepth}{0}
% Redefines (sub)paragraphs to behave more like sections
\ifx\paragraph\undefined\else
\let\oldparagraph\paragraph
\renewcommand{\paragraph}[1]{\oldparagraph{#1}\mbox{}}
\fi
\ifx\subparagraph\undefined\else
\let\oldsubparagraph\subparagraph
\renewcommand{\subparagraph}[1]{\oldsubparagraph{#1}\mbox{}}
\fi

%%% Use protect on footnotes to avoid problems with footnotes in titles
\let\rmarkdownfootnote\footnote%
\def\footnote{\protect\rmarkdownfootnote}

%%% Change title format to be more compact
\usepackage{titling}

% Create subtitle command for use in maketitle
\newcommand{\subtitle}[1]{
  \posttitle{
    \begin{center}\large#1\end{center}
    }
}

\setlength{\droptitle}{-2em}
  \title{Tarea 2}
  \pretitle{\vspace{\droptitle}\centering\huge}
  \posttitle{\par}
  \author{Eduardo Gamboa \\ Manuel Marín \\ Megan Alexa Mora \\ Francisco Sánchez}
  \preauthor{\centering\large\emph}
  \postauthor{\par}
  \date{}
  \predate{}\postdate{}


\begin{document}
\maketitle

\section{Carga de datos}\label{carga-de-datos}

Se procede a realizar la carga de datos en R, a fin de poder realizar
los cálculos necesarios sobre los mismos

\begin{Shaded}
\begin{Highlighting}[]
\NormalTok{valores <-}\StringTok{ }\KeywordTok{c}\NormalTok{(}\DecValTok{7}\NormalTok{,}\DecValTok{7}\NormalTok{,}\DecValTok{15}\NormalTok{,}\DecValTok{11}\NormalTok{,}\DecValTok{9}\NormalTok{,}\DecValTok{12}\NormalTok{,}\DecValTok{17}\NormalTok{,}\DecValTok{12}\NormalTok{,}\DecValTok{18}\NormalTok{,}\DecValTok{18}\NormalTok{,}\DecValTok{14}\NormalTok{,}\DecValTok{18}\NormalTok{,}\DecValTok{18}\NormalTok{,}\DecValTok{19}\NormalTok{,}\DecValTok{19}\NormalTok{,}\DecValTok{19}\NormalTok{,}\DecValTok{25}\NormalTok{,}\DecValTok{22}\NormalTok{,}\DecValTok{19}\NormalTok{,}\DecValTok{23}\NormalTok{,}\DecValTok{7}\NormalTok{,}\DecValTok{10}\NormalTok{,}\DecValTok{11}\NormalTok{,}\DecValTok{15}\NormalTok{,}\DecValTok{11}\NormalTok{)}
\NormalTok{porcentajes <-}\StringTok{ }\KeywordTok{as.factor}\NormalTok{(}\KeywordTok{c}\NormalTok{(}\KeywordTok{rep}\NormalTok{(}\KeywordTok{c}\NormalTok{(}\StringTok{"15%"}\NormalTok{, }\StringTok{"20%"}\NormalTok{, }\StringTok{"25%"}\NormalTok{, }\StringTok{"30%"}\NormalTok{, }\StringTok{"35%"}\NormalTok{), }\DataTypeTok{each=}\DecValTok{5}\NormalTok{)))}
\KeywordTok{summary}\NormalTok{(valores)}
\end{Highlighting}
\end{Shaded}

\begin{verbatim}
##    Min. 1st Qu.  Median    Mean 3rd Qu.    Max. 
##    7.00   11.00   15.00   15.04   19.00   25.00
\end{verbatim}

\begin{Shaded}
\begin{Highlighting}[]
\KeywordTok{summary}\NormalTok{(porcentajes)}
\end{Highlighting}
\end{Shaded}

\begin{verbatim}
## 15% 20% 25% 30% 35% 
##   5   5   5   5   5
\end{verbatim}

Para verificacion, se calculan los totales por tratamiento, los
promedios por tratamiento, el gran totaly el gran promedio:

\begin{itemize}
\tightlist
\item
  Promedio por tratamiento:
\end{itemize}

\begin{Shaded}
\begin{Highlighting}[]
\KeywordTok{tapply}\NormalTok{(valores, porcentajes, mean)}
\end{Highlighting}
\end{Shaded}

\begin{verbatim}
##  15%  20%  25%  30%  35% 
##  9.8 15.4 17.6 21.6 10.8
\end{verbatim}

\begin{itemize}
\tightlist
\item
  Totales por tratamiento
\end{itemize}

\begin{Shaded}
\begin{Highlighting}[]
\KeywordTok{tapply}\NormalTok{(valores, porcentajes, sum)}
\end{Highlighting}
\end{Shaded}

\begin{verbatim}
## 15% 20% 25% 30% 35% 
##  49  77  88 108  54
\end{verbatim}

\begin{itemize}
\tightlist
\item
  Gran Total
\end{itemize}

\begin{Shaded}
\begin{Highlighting}[]
\KeywordTok{sum}\NormalTok{(valores)}
\end{Highlighting}
\end{Shaded}

\begin{verbatim}
## [1] 376
\end{verbatim}

\begin{itemize}
\tightlist
\item
  Gran Promedio
\end{itemize}

\begin{Shaded}
\begin{Highlighting}[]
\KeywordTok{mean}\NormalTok{(valores)}
\end{Highlighting}
\end{Shaded}

\begin{verbatim}
## [1] 15.04
\end{verbatim}

Además, se grafincan los las muestras para ver su comportamiento:

\begin{Shaded}
\begin{Highlighting}[]
\KeywordTok{boxplot}\NormalTok{(valores }\OperatorTok{~}\StringTok{ }\NormalTok{porcentajes)}
\end{Highlighting}
\end{Shaded}

\includegraphics{Tarea-2_files/figure-latex/unnamed-chunk-6-1.pdf}

Es claro ver que hay algunas variables atípicas en algunos de los
tratamientos. Casos de esto se puedne ver en el tratamiento de 35\%, y
en el que tiene el tratamiento de 25\%. Esto debido a que los valores
mas altos o bajos distan mucho respecto al promedio obtenido. Además se
denota cierta tendencia a que la mayoría de los datos donde la fibra
tenia 30\% de algodón fue la que más presión resistió, pues la mayoría
de sus observaciones estan por encima de los otros tratamientos,
apreciable por donde se encuentra la caja en el gráfico anterior.

\section{Calculo del Anova}\label{calculo-del-anova}

Se calcula el Anova en R:

\begin{Shaded}
\begin{Highlighting}[]
\NormalTok{fm =}\StringTok{ }\KeywordTok{aov}\NormalTok{(}\KeywordTok{lm}\NormalTok{(valores }\OperatorTok{~}\StringTok{ }\NormalTok{porcentajes))}
\KeywordTok{summary}\NormalTok{(fm)}
\end{Highlighting}
\end{Shaded}

\begin{verbatim}
##             Df Sum Sq Mean Sq F value   Pr(>F)    
## porcentajes  4  475.8  118.94   14.76 9.13e-06 ***
## Residuals   20  161.2    8.06                     
## ---
## Signif. codes:  0 '***' 0.001 '**' 0.01 '*' 0.05 '.' 0.1 ' ' 1
\end{verbatim}

Según los resultados obtenidos, el valor p de la prueba de hipótesis
realizada, muestra una significacia menor que 0.05, lo que indica que
existe evidencia para rechazar la hipótesis nula. En otras palabras, se
atribuye al diferencia de los valores a los cambios del tratamiento, y
no por las muestras tomadas durante los experimentos.

\section{Validación de los
tratamientos}\label{validacion-de-los-tratamientos}

El ANOVA indica, por medio de su significancia, que existe al menos un
par que es diferente al resto, sin embargo, el modelo no indica cual de
ellos es.

\section*{Referencias}\label{referencias}
\addcontentsline{toc}{section}{Referencias}

\hypertarget{refs}{}
\hypertarget{ref-anova-r}{}
García, Francisco. n.d. ``Análisis de La Varianza (ANOVA) Con R.''
\url{https://biocosas.github.io/R/050_anova.html}.


\end{document}
